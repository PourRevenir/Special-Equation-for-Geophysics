\documentclass[12pt]{ctexart}
    %%% Document Settings %%%%
%\usepackage[utf8]{inputenc}

\usepackage[
    twoside,
    top=1in,
    bottom=0.75in,
    inner=0.5in,
    outer=0.5in,
]{geometry}
\pagestyle{myheadings}
\usepackage[dvipsnames,svgnames]{xcolor}

%%%% Additional Commands to Load %%%%
\usepackage{tcolorbox}
\tcbuselibrary{skins}
%\usepackage{minted}
\usepackage{color}
\usepackage{tikz}
\usetikzlibrary{calc}
\usepackage{tabularx,colortbl}
\usepackage{amsfonts,amsmath,amssymb}
\usepackage{titling}
\usepackage{mathrsfs}
\usepackage{calc}
\usepackage{subcaption}

\usepackage{listings}
%\usepackage{newtxtext}
\usepackage[strict]{changepage} 
\usepackage{framed}
\definecolor{formalshade}{rgb}{0.95,0.95,1}

%%%% Commands to Define Homework Boxes %%%%
%%%% Box Definition %%%%
\newtcolorbox{prob}[1]{
% Set box style
    sidebyside,
    sidebyside align=top,
% Dimensions and layout
    width=\textwidth,
    toptitle=2.5pt,
    bottomtitle=2.5pt,
    righthand width=0.20\textwidth,
% Coloring
    colbacktitle=gray!30,
    coltitle=black,
    colback=white,
    colframe=black,
% Title formatting
    title={
        #1 \hfill Grade:\phantom{WWWW}
    },
    fonttitle=\large\bfseries
}

%%%% Environment Definition %%%%
\newenvironment{problem}[1]{
    \begin{prob}{#1}
}
{
    \tcblower
    \centering
    \textit{\scriptsize\bfseries Faculty Comments}
    \vspace{\baselineskip}
    \end{prob}
}

\newenvironment{formal}{%
\def\FrameCommand{%
\hspace{1pt}%
{\color{DarkBlue}\vrule width 2pt}%
{\color{formalshade}\vrule width 4pt}%
\colorbox{formalshade}%
}%
\MakeFramed{\advance\hsize-\width\FrameRestore}%
\noindent\hspace{-4.55pt}% disable indenting first paragraph
\begin{adjustwidth}{}{7pt}%
\vspace{2pt}\vspace{2pt}%
}
{%
\vspace{2pt}\end{adjustwidth}\endMakeFramed%
}

    \title{特殊方程作业2}
    \author{地物2201班\ 杨曜堃}
    \date{\today}
%%% document
\begin{document}

% Format Running Header
    \markboth{\theauthor}{\thetitle}
    \maketitle
    求解满足下列边界条件及初始条件的弦振动方程。
    \begin{description}
        \item[问题1]$$\begin{cases}
            \dfrac{\partial^2u}{\partial t^2}=\dfrac{\partial^2u}{\partial x^2},&\quad 0<x<1,t>0\\
            u|_{x=0}=0,\ u|_{x=1}=0,&\quad t\geqslant 0\\
            u|_{t=0}=\sin \pi x,\ \dfrac{\partial u}{\partial t}|_{t=0}=0,&\quad 0\leqslant x\leqslant 1
        \end{cases}$$
        \item[问题2]$$\begin{cases}
            \dfrac{\partial^2u}{\partial t^2}=4\dfrac{\partial^2u}{\partial x^2},&\quad 0<x<1,t>0\\
            u|_{x=0}=0,\ u|_{x=1}=0,&\quad t\geqslant 0\\
            u|_{t=0}=\sin 2\pi x,\ \dfrac{\partial u}{\partial t}|_{t=0}=\sin 3\pi x,&\quad 0\leqslant x\leqslant 1
        \end{cases}$$
        \item[问题3]$$\begin{cases}
            \dfrac{\partial^2u}{\partial t^2}=\dfrac{\partial^2u}{\partial x^2},&\quad 0<x<1,t>0\\
            u|_{x=0}=0,\ u|_{x=1}=0,&\quad t\geqslant 0\\
            u|_{t=0}=\sin \pi x +3\sin 2\pi x-\sin 5\pi x,\ \dfrac{\partial u}{\partial t}|_{t=0}=0,&\quad 0\leqslant x\leqslant 1
        \end{cases}$$
    \end{description}
    \begin{problem}{问题\#1}
        分离变量法,设$u(x,t)=X(x)T(t)$,代入偏微分方程可得
        $$
        \dfrac{X''(x)}{X(x)}=\dfrac{T''(t)}{T(t)}=-\lambda
        $$
        代入边界条件,得到常微分方程的边值问题
        $$
        \begin{cases}
            X''(t)+\lambda X(t)=0\\
            X(0)=X(1)=0
        \end{cases}
        $$
        采用本征值法讨论,排除零解情况,得到在$\lambda >0$时
        $$
        X(x)=A\cos\sqrt{\lambda}x+B\sin\sqrt{\lambda}x
        $$
        代入边界条件得到
        $$
        X(0)=A=0,\ X(1)=B\sin\sqrt{-\lambda}=0
        $$
        由于$B$不能为0,所以$\sin\sqrt{\lambda}=0$,解出
        $$
        \lambda_n=(n\pi)^2,\ X_n(x)=B\sin n\pi x,\ n=1,2,\cdots
        $$
        进一步可以解出
        $$
        T_n(t)=C_n\cos n\pi t+D_n\sin n\pi t,\ n=1,2,\cdots
        $$
        得到满足条件的一组特解
        $$
        u_n(x,t)=(C_n\cos n\pi t+D_n\sin n\pi t)\sin n\pi x,\ n=1,2,\cdots
        $$
        代入初始条件
        $$
        \begin{cases}
            u|_{t=0}=\sum C_n\sin n\pi x=\sin \pi x,\\
            \dfrac{\partial u}{\partial t}|_{t=0} =n\pi\sum D_n\sin n\pi x=0 
        \end{cases}
        ,n=1,2,\cdots
        $$
        因此我们取$C_1=1$,$D_n=0$,得到形式解 
        $$
        u(x,t)=\cos\pi t\sin \pi x
        $$
    \end{problem}
    \begin{problem}{问题\#2}
        分离变量法,设$u(x,t)=X(x)T(t)$,代入偏微分方程可得
        $$
        \dfrac{X''(x)}{X(x)}=\dfrac{1}{4}\dfrac{T''(t)}{T(t)}=-\lambda
        $$
        代入边界条件,得到常微分方程的边值问题
        $$
        \begin{cases}
            X''(t)+\lambda X(t)=0\\
            X(0)=X(1)=0
        \end{cases}
        $$
        采用本征值法讨论,排除零解情况,解出本征值和本征函数
        $$
        \lambda_n=(n\pi)^2,\ X_n(x)=\sin n\pi x,\ n=1,2,\cdots
        $$
        进一步可以解出
        $$
        T_n(t)=C_n\cos 2n\pi t+D_n\sin 2n\pi t,\ n=1,2,\cdots
        $$
        得到满足条件的一组特解
        $$
        u_n(x,t)=(C_n\cos 2n\pi t+D_n\sin 2n\pi t)\sin n\pi x,\ n=1,2,\cdots
        $$
        代入初始条件
        $$
        \begin{cases}
            u|_{t=0}=\sum C_n\sin n\pi x=\sin 2\pi x,\\
            \dfrac{\partial u}{\partial t}|_{t=0}=2n\pi\sum D_n\sin n\pi x=\sin 3\pi x
        \end{cases}
        ,n=1,2,\cdots
        $$
        取$C_2=1$,$D_3=\dfrac{1}{6\pi}$,得到形式解
        $$
        u(x,t)=\cos4\pi t\sin2\pi x+\dfrac{1}{6\pi}\sin6\pi t\sin3\pi x
        $$
    \end{problem}
    \begin{problem}{问题\#3}
        分离变量法,设$u(x,t)=X(x)T(t)$,代入偏微分方程可得
        $$
        \dfrac{X''(x)}{X(x)}=\dfrac{T''(t)}{T(t)}=-\lambda
        $$
        采用本征值法讨论,并代入边界条件,解出
        $$
        X_n(x)=\sin n\pi x,\ n=1,2,\cdots
        $$
        $$
        T_n(t)=C_n\cos n\pi t+D_n\sin n\pi t,\ n=1,2,\cdots
        $$
        得到满足条件的一组特解
        $$
        u_n(x,t)=(C_n\cos n\pi t+D_n\sin n\pi t)\sin n\pi x,\ n=1,2,\cdots
        $$
        代入初始条件
        $$
        \begin{cases}
            u|_{t=0}=\sum C_n\sin n\pi x=\sin\pi x+3\sin2\pi x-\sin5\pi x\\
            \dfrac{\partial u}{\partial t}|_{t=0}=n\pi\sum D_n\sin n\pi x=0
        \end{cases}
        ,n=1,2,\cdots
        $$
        我们取$C_1=1$,$C_2=3$,$C_5=-1$,$D_n=0$,得到形式解
        $$
        u(x,t)=\cos\pi t\sin\pi x+3\cos2\pi t\sin2\pi x-\cos5\pi t\sin5\pi x
        $$
    \end{problem}
\end{document}