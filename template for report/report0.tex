\documentclass[12pt]{article}

\usepackage[a4paper]{geometry}
\geometry{left=2.0cm,right=2.0cm,top=2.5cm,bottom=2.5cm}

\usepackage{ctex}

\usepackage[dvipsnames,svgnames]{xcolor}
\usepackage{tcolorbox}
\tcbuselibrary{skins}
\usepackage{minted}

\usepackage{color}
\usepackage{tikz}
\usetikzlibrary{calc}
\usepackage{tabularx,colortbl}
\usepackage{amsfonts,amsmath,amssymb}
\usepackage{titling}
\usepackage{mathrsfs}
\usepackage{calc}
\usepackage{listings}

\usepackage[strict]{changepage} 
\usepackage{framed}
\definecolor{formalshade}{rgb}{0.95,0.95,1}
\usepackage{float}

\newenvironment{problem}[1]{
    \begin{prob}{#1}
}
{
    \tcblower
    \centering
    \textit{\scriptsize\bfseries Faculty Comments}
    \vspace{\baselineskip}
    \end{prob}
}

% 在此输入基本信息
\newcommand{\experiName}{(实验名称)}
\newcommand{\advisorName}{(指导教师姓名)}
\newcommand{\studentName}{(学生姓名)}
\newcommand{\studentNum}{(学号)}
\newcommand{\classNum}{(班级)}

\begin{document}

\begin{center}
    ~\\
    ~\\
    \huge \bf 《地球物理特殊方程》实验报告
\end{center}

\vspace{2.5cm}
\begin{figure}[htbp]
    \centering
    \includegraphics[width=5.5cm]{Fig/logo.png}
\end{figure}

\vspace{4cm}
\begin{center}
    \Large \bf 班\qquad 级:\underline{\makebox[12em][c]{\kaishu \classNum}}

    \Large \bf 姓\qquad 名:\underline{\makebox[12em][c]{\kaishu \studentName}}

    \Large \bf 指导老师:\underline{\makebox[12em][c]{\kaishu \advisorName}}
\end{center}

\clearpage

\begin{center}
    \huge 实验一 \quad (实验名称)
\end{center}
\vspace{0.2cm}
\section{实验内容}

\section{实验目的}

\section{实验过程及结果}

\section{实验总结及收获}

\end{document}